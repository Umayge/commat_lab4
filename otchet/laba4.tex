\documentclass[12pt,a4paper]{article}
\usepackage[14pt]{extsizes}
\usepackage[utf8]{inputenc}
\usepackage{amsmath}
\usepackage{amsfonts}
\usepackage{amssymb}
\usepackage{tabularx}
\usepackage{multirow}
\newcommand{\RomanNumeralCaps}[1]
{\MakeUppercase{\romannumeral #1}}
\usepackage{cmap} % для кодировки шрифтов в pdf
\usepackage[T2A]{fontenc}
\usepackage[russian]{babel}
\usepackage{graphicx} % для вставки картинок
\usepackage{amssymb,amsfonts,amsmath,amsthm} % математические дополнения от АМС
\usepackage{indentfirst} % отделять первую строку раздела абзацным отступом тоже
\usepackage[outdir=./]{epstopdf}
\usepackage{listings}
\usepackage{color} %red, green, blue, yellow, cyan, magenta, black, white
\definecolor{mygreen}{RGB}{28,172,0} % color values Red, Green, Blue
\definecolor{mylilas}{RGB}{170,55,241}
% Поля
\usepackage{geometry}
\geometry{left=2.5cm}
\geometry{right=1.5cm}
\geometry{top=1.5cm}
\geometry{bottom=2cm}

%%%%%%%%%%%%%%%%%%%%%%%%%%%%%%%    

\linespread{1.5} % полуторный интервал
\renewcommand{\rmdefault}{ftm} % Times New Roman
%\renewcommand{\rmdefault}{PT-Astra-Serif_Regular}
%\usefont{T2A}{PT-Astra-Serif_Regular}{m}{it}
\frenchspacing

% подключаем hyperref (для ссылок внутри  pdf)
\usepackage[unicode, pdftex]{hyperref}

% коррекция подрисуночной подписи
\usepackage{caption}
\captionsetup[figure]{labelsep=period} %Точка
%\captionsetup[figure]{labelsep=space}  %Пробел
%\captionsetup[figure]{labelsep=endash}  %Новый страндарт
\usepackage[figurename=Рисунок]{caption}
\usepackage{longtable}


\begin{document}
	
	\begin{titlepage}
		
		\begin{center}
			САНКТ-ПЕТЕРБУРГСКИЙ ГОСУДАРСТВЕННЫЙ ЭЛЕКТРОТЕХНИЧЕСКИЙ УНИВЕРСИТЕТ «ЛЭТИ» ИМ. В.И. УЛЬЯНОВА (ЛЕНИНА)\\
			\vspace{0.1cm}
			
			
			
			
		\end{center}
	\vspace{2cm}
	\begin{center}
		\begin{large}
			Факультет компьютерных технологий и информатики\\
			Кафедра Алгоритмической математики\\
		\end{large}
	\end{center}
		
		\vspace{5cm}
		\begin{center}
			\begin{large}
				Лабораторная работа 4 \\
				"Численное интегрирование"
			\end{large}
		\end{center}
		
		\vspace{5cm}
		
		\hspace{5cm} Студент гр.  0307 \hrulefill Латин Я.М.
		
		\vspace{0.5cm}
		\hspace{5cm} Преподаватель \hrulefill Солнышкин С.Н. \\
		
		
		\vfill
		\begin{center}
			Санкт-Петербург\\
			2022
		\end{center}
		
	\end{titlepage}
	\tableofcontents
	\newpage
	\section{Задание(Вариант 191)}
	Вычислить приближённо интегралы:\\
	1).$\displaystyle\int\limits_0^6 \dfrac{x^2-3x-5}{x^2+2x+4}\,dx$
	\hspace{4cm}
	2).$\displaystyle\int\limits_0^5 \sqrt{\sin\dfrac{x}{4}}\,dx$\\
	3). $\displaystyle\int\limits_0^{2\pi} \sqrt{7-5\cos x}\,dx$
	\hspace{3.75cm}
	4).$\displaystyle \int \limits_0^\infty \dfrac{5+3x}{1+2x^5+4x^6}\,dx$
	
	\section{Вычисления}
	\subsection{$\displaystyle\int\limits_0^6 \dfrac{x^2-3x-5}{x^2+2x+4}\,dx=-2.270595273947305$}
	
	\subsubsection*{Формула левых прямоугольников:}
	$\displaystyle\int \limits_{x_0}^{x_1} f(x)\,dx = h(f_0+f_1+...+f_{n-1}),m=1$
	
% Please add the following required packages to your document preamble:
% \usepackage{graphicx}
\begin{table}[!h]
	\resizebox{\columnwidth}{!}{%
		\begin{tabular}{|c|c|c|c|c|c|c|}
			\hline
			n & $I_h$ & $I_0-I_h$ & $I_h-I_{2h}$ & $\dfrac{I_h-I_{2h}}{2^m-1}$ & $\dfrac{I_{2h}-I_{4h}}{I_h-I_{2h}}$ & $\log_2\left(\dfrac{I_{2h}-I_{4h}}{I_h-I_{2h}}\right)$ \\ \hline
			2    & -4.539473684210526 & 2.268878410263221    & 0                    & 0                    & 0                 & 0                  \\ \hline
			4    & -3.366465149359886 & 1.095869875412581    & 1.17300853485064     & 1.17300853485064     & 0                 & 0                  \\ \hline
			8    & -2.823077734346899 & 0.5524824603995939   & 0.5433874150129872   & 0.5433874150129872   & 2.158696544016582 & 1.110160452618335  \\ \hline
			16   & -2.549188446981114 & 0.2785931730338094   & 0.2738892873657846   & 0.2738892873657846   & 1.983967391493054 & 0.9883883137702176 \\ \hline
			32   & -2.41054712050002  & 0.1399518465527154   & 0.138641326481094    & 0.138641326481094    & 1.975524140726783 & 0.9822354760265514 \\ \hline
			64   & -2.34073893590043  & 0.07014366195312549  & 0.06980818459958993  & 0.06980818459958993  & 1.986032544411825 & 0.989889263992599  \\ \hline
			128  & -2.305709280228379 & 0.03511400628107353  & 0.03502965567205196  & 0.03502965567205196  & 1.992831024464955 & 0.9948193867944236 \\ \hline
			256  & -2.288162835881936 & 0.01756756193463094  & 0.01754644434644259  & 0.01754644434644259  & 1.99639624874506  & 0.9973980981115383 \\ \hline
			512  & -2.279381695547521 & 0.00878642160021581  & 0.008781140334415127 & 0.008781140334415127 & 1.998196552863915 & 0.9986985009957414 \\ \hline
			1024 & -2.274989144964017 & 0.004393871016711959 & 0.004392550583503851 & 0.004392550583503851 & 1.999098284125071 & 0.9993494028149986 \\ \hline
		\end{tabular}%
	}
\end{table}

\subsubsection*{Формула правых прямоугольников:}
$\displaystyle\int \limits_{x_0}^{x_1} f(x)\,dx = h(f_1+f_2+...+f_{n}),m=1$
\newpage
\begin{table}[!h]
	\resizebox{\columnwidth}{!}{%
		\begin{tabular}{|c|c|c|c|c|c|c|}
			\hline
			n &
			$I_h$ &
			$I_0-I_h$ &
			$I_h-I_{2h}$ &
			$\dfrac{I_h-I_{2h}}{2^m-1}$ &
			$\dfrac{I_{2h}-I_{4h}}{I_h-I_{2h}}$ &
			$\log_2\left(\dfrac{I_{2h}-I_{4h}}{I_h-I_{2h}}\right)$ \\ \hline
			2   & -0.03947368421052627 & -2.231121589736779   & 0                     & 0                     & 0                 & 0                  \\ \hline
			4   & -1.116465149359886   & -1.154130124587419   & -1.07699146514936     & -1.07699146514936     & 0                 & 0                  \\ \hline
			8   & -1.698077734346899   & -0.5725175396004061  & -0.5816125849870128   & -0.5816125849870128   & 1.851733426939874 & 0.8888764251032127 \\ \hline
			16  & -1.986688446981114   & -0.2839068269661906  & -0.2886107126342154   & -0.2886107126342154   & 2.015214818876621 & 1.010933636077808  \\ \hline
			32  & -2.12929712050002    & -0.1412981534472846  & -0.142608673518906    & -0.142608673518906    & 2.023794945375139 & 1.017063120923044  \\ \hline
			64  & -2.20011393590043    & -0.07048133804687451 & -0.07081681540041007  & -0.07081681540041007  & 2.013768519702176 & 1.009897856795099  \\ \hline
			128 & -2.235396780228379   & -0.03519849371892647 & -0.03528284432794804  & -0.03528284432794804  & 2.007117531176903 & 1.005125099351972  \\ \hline
			256 & -2.253006585881936   & -0.01758868806536906 & -0.01760980565355741  & -0.01760980565355741  & 2.003590784706954 & 1.002587881208276  \\ \hline
			512 & -2.261803570547521   & -0.00879170339978419 & -0.008796984665584873 & -0.008796984665584873 & 2.00180019893062  & 1.001297984964717  \\ \hline
			1024 &
			-2.266200082464017 &
			-0.004395191483288041 &
			-0.004396511916496149 &
			-0.004396511916496149 &
			2.000900903413389 &
			1.000649718121073 \\ \hline
		\end{tabular}%
	}
\end{table}
	
\subsubsection*{Формула центральных прямоугольников:}
$\displaystyle\int \limits_{x_0}^{x_1} f(x)\,dx = h(f_{\frac{1}{2}}+f_{\frac{3}{2}}+...+f_{n-\frac{1}{2}}),m=2$
% Please add the following required packages to your document preamble:
% \usepackage{graphicx}
\begin{table}[!h]
	\resizebox{\columnwidth}{!}{%
		\begin{tabular}{|c|c|c|c|c|c|c|}
			\hline
			n & $I_h$ & $I_0-I_h$ & $I_h-I_{2h}$ & $\dfrac{I_h-I_{2h}}{2^m-1}$ & $\dfrac{I_{2h}-I_{4h}}{I_h-I_{2h}}$ & $\log_2\left(\dfrac{I_{2h}-I_{4h}}{I_h-I_{2h}}\right)$ \\ \hline
			2    & -2.193456614509246 & -0.07713865943805853  & 0                     & 0                     & 0                  & 0                  \\ \hline
			4    & -2.279690319333912 & 0.00909504538660677   & -0.0862337048246653   & -0.02874456827488843  & 0                  & 0                  \\ \hline
			8    & -2.275299159615329 & 0.004703885668024377  & 0.004391159718582394  & 0.001463719906194131  & -19.63802511207775 & 4.295577947947852  \\ \hline
			16   & -2.271905794018926 & 0.001310520071621024  & 0.003393365596403353  & 0.001131121865467784  & 1.294042623416884  & 0.3718851378709352 \\ \hline
			32   & -2.270930751300841 & 0.0003354773535360067 & 0.0009750427180850174 & 0.0003250142393616725 & 3.480222490218601  & 1.799179540348291  \\ \hline
			64   & -2.270679624556327 & 8.435060902156621e-05 & 0.0002511267445144405 & 8.370891483814684e-05 & 3.882671755930598  & 1.957049745848512  \\ \hline
			128  & -2.270616391535492 & 2.111758818745813e-05 & 6.323302083410809e-05 & 2.107767361136936e-05 & 3.971449429456673  & 1.989665632823965  \\ \hline
			256  & -2.270600555213107 & 5.281265802015156e-06 & 1.583632238544297e-05 & 5.27877412848099e-06  & 3.99291068311624   & 1.997440800879939  \\ \hline
			512  & -2.270596594380517 & 1.320433212104177e-06 & 3.960832589910979e-06 & 1.320277529970326e-06 & 3.998230681544381  & 1.999361712082348  \\ \hline
			1024 & -2.270595604062906 & 3.301156010770967e-07 & 9.903176110270806e-07 & 3.301058703423602e-07 & 3.99955786487843   & 1.999840524649308  \\ \hline
		\end{tabular}%
	}
\end{table}

\subsubsection*{Формула трапеций:}
$\displaystyle\int \limits_{x_0}^{x_1} f(x)\,dx = \dfrac{h}{2}(f_{0}+2f_{1}+...+2f_{n-1}+f_n),m=2$

% Please add the following required packages to your document preamble:
% \usepackage{graphicx}
\begin{table}[!h]
	\resizebox{\columnwidth}{!}{%
		\begin{tabular}{|c|c|c|c|c|c|c|}
			\hline
			n & $I_h$ & $I_0-I_h$ & $I_h-I_{2h}$ & $\dfrac{I_h-I_{2h}}{2^m-1}$ & $\dfrac{I_{2h}-I_{4h}}{I_h-I_{2h}}$ & $\log_2\left(\dfrac{I_{2h}-I_{4h}}{I_h-I_{2h}}\right)$ \\ \hline
			2    & -2.289473684210526 & 0.01887841026322112    & 0                      & 0                      & 0                  & 0                 \\ \hline
			4    & -2.241465149359886 & -0.02913012458741893   & 0.04800853485064005    & 0.01600284495021335    & 0                  & 0                 \\ \hline
			8    & -2.260577734346899 & -0.01001753960040608   & -0.01911258498701285   & -0.006370861662337616  & -2.511880778202542 & 1.328767990989993 \\ \hline
			16   & -2.267938446981114 & -0.002656826966190629  & -0.007360712634215449  & -0.002453570878071817  & 2.596567198965238  & 1.376605562748794 \\ \hline
			32   & -2.26992212050002  & -0.0006731534472845802 & -0.001983673518906048  & -0.0006612245063020161 & 3.710647222973828  & 1.891670848284495 \\ \hline
			64   & -2.27042643590043  & -0.0001688380468745088 & -0.0005043154004100714 & -0.0001681051334700238 & 3.933398657453399  & 1.975776413584211 \\ \hline
			128  & -2.270553030228379 & -4.224371892647127e-05 & -0.0001265943279480375 & -4.21981093160125e-05  & 3.983712450506275  & 1.994113515578606 \\ \hline
			256  & -2.270584710881936 & -1.056306536906249e-05 & -3.168065355740879e-05 & -1.05602178524696e-05  & 3.995950642831115  & 1.998538763364928 \\ \hline
			512  & -2.270592633047521 & -2.640899784189799e-06 & -7.922165584872687e-06 & -2.640721861624229e-06 & 3.998989066563157  & 1.999635336752774 \\ \hline
			1024 & -2.270594613714017 & -6.602332880412121e-07 & -1.980666496148586e-06 & -6.602221653828622e-07 & 3.999747357910768  & 1.999908875749928 \\ \hline
		\end{tabular}%
	}
\end{table}
\subsubsection*{Формула Симпсона:}
$\displaystyle\int \limits_{x_0}^{x_1} f(x)\,dx = \dfrac{h}{3}(f_{0}+4f_{1}+2f_2+...+4f_{n-1}+f_n),m=4$
% Please add the following required packages to your document preamble:
% \usepackage{graphicx}
% Please add the following required packages to your document preamble:
% \usepackage{graphicx}
\begin{table}[!h]
	\resizebox{\columnwidth}{!}{%
		\begin{tabular}{|c|c|c|c|c|c|c|}
			\hline
			n & $I_h$ & $I_0-I_h$ & $I_h-I_{2h}$ & $\dfrac{I_h-I_{2h}}{2^m-1}$ & $\dfrac{I_{2h}-I_{4h}}{I_h-I_{2h}}$ & $\log_2\left(\dfrac{I_{2h}-I_{4h}}{I_h-I_{2h}}\right)$ \\ \hline
			2    & -2.052631578947368 & -0.2179636949999368    & 0                      & 0                      & 0                 & 0                 \\ \hline
			4    & -2.225462304409673 & -0.04513296953763213   & -0.1728307254623047    & -0.01152204836415365   & 0                 & 0                 \\ \hline
			8    & -2.266948596009236 & -0.00364667793806861   & -0.04148629159956352   & -0.002765752773304235  & 4.165971910203782 & 2.058653111787027 \\ \hline
			16   & -2.270392017859186 & -0.0002032560881191081 & -0.003443421849949502  & -0.0002295614566633001 & 12.04798407147585 & 3.59071986263783  \\ \hline
			32   & -2.270583345006322 & -1.192894098300812e-05 & -0.0001913271471361    & -1.275514314240667e-05 & 17.99756020770034 & 4.169729439507748 \\ \hline
			64   & -2.270594541033901 & -7.329134041889063e-07 & -1.119602757881921e-05 & -7.464018385879474e-07 & 17.08884207270578 & 4.09498273926473  \\ \hline
			128  & -2.270595228337695 & -4.560961031074839e-08 & -6.873037938781579e-07 & -4.582025292521053e-08 & 16.28977997581663 & 4.025895212566511 \\ \hline
			256  & -2.270595271099788 & -2.847516888948576e-09 & -4.276209342179982e-08 & -2.850806228119988e-09 & 16.07273495941093 & 4.006543535826699 \\ \hline
			512  & -2.270595273769383 & -1.779216773911685e-10 & -2.669595211557407e-09 & -1.779730141038271e-10 & 16.01819378333877 & 4.001639573050281 \\ \hline
			1024 & -2.270595273936185 & -1.111999381464557e-11 & -1.668016835765229e-10 & -1.112011223843486e-11 & 16.00460591473999 & 4.000415248381294 \\ \hline
		\end{tabular}%
	}
\end{table}

\subsubsection*{Формула Гауса:}
$\displaystyle\int \limits_{x_0}^{x_1} f(x)\,dx = \dfrac{h}{18}(f_{k_{-1}}+2f_{k_0}+f_{k_1}),m=6$
% Please add the following required packages to your document preamble:
% \usepackage{graphicx}
\begin{table}[!h]
	\resizebox{\columnwidth}{!}{%
		\begin{tabular}{|c|c|c|c|c|c|c|}
			\hline
			n & $I_h$ & $I_0-I_h$ & $I_h-I_{2h}$ & $\dfrac{I_h-I_{2h}}{2^m-1}$ & $\dfrac{I_{2h}-I_{4h}}{I_h-I_{2h}}$ & $\log_2\left(\dfrac{I_{2h}-I_{4h}}{I_h-I_{2h}}\right)$ \\ \hline
			2    & -2.271547926155566 & 0.0009526522082610178  & 0                      & 0                      & 0                  & 0                 \\ \hline
			4    & -2.270571317949083 & -2.39559982215809e-05  & 0.0009766082064825987  & 1.550171756321585e-05  & 0                  & 0                 \\ \hline
			8    & -2.27059448265613  & -7.912911748064744e-07 & -2.316470704677442e-05 & -3.676937626472131e-07 & -42.15931608850572 & 5.397779558548351 \\ \hline
			16   & -2.270595260998149 & -1.294915641381067e-08 & -7.783420183926637e-07 & -1.235463521258196e-08 & 29.76160415264658  & 4.895380384797567 \\ \hline
			32   & -2.27059527374508  & -2.02224903489423e-10  & -1.274693151032125e-08 & -2.023322461955753e-10 & 61.0611281438546   & 5.932182339402726 \\ \hline
			64   & -2.270595273944147 & -3.158362460453645e-12 & -1.990665410289694e-10 & -3.159786365539196e-12 & 64.03352087413944  & 6.000755433415594 \\ \hline
			128  & -2.270595273947256 & -4.884981308350689e-14 & -3.109512647370138e-12 & -4.935734360904981e-14 & 64.01856612396458  & 6.000418458915312 \\ \hline
			256  & -2.270595273947304 & -1.332267629550188e-15 & -4.75175454539567e-14  & -7.542467532374079e-16 & 65.4392523364486   & 6.032084360027357 \\ \hline
			512  & -2.270595273947305 & 0                      & -1.332267629550188e-15 & -2.114710523095536e-17 & 35.66666666666666  & 5.15650448567999  \\ \hline
			1024 & -2.270595273947305 & -4.440892098500626e-16 & 4.440892098500626e-16  & 7.049035076985121e-18  & -3                 & 1.584962500721156 \\ \hline
		\end{tabular}%
	}
\end{table}

\subsection{$\displaystyle\int\limits_0^5 \sqrt{\sin\dfrac{x}{4}}\,dx=3.52040903029251$}
Так как подынтегральная функция не гладкая, правило Рунге не работает. Поэтому необходимо произвести замену переменной.\\ 

$\displaystyle\int\limits_0^5 \sqrt{\sin\dfrac{x}{4}}\,dx=\begin{bmatrix}x=t^2&0=0\\dx=2tdt&5=\sqrt{5}\end{bmatrix}=\int\limits_0^{\sqrt{5}} 2t\sqrt{\sin\dfrac{t^2}{4}}\,dt=3.52040903029251\\$

\subsubsection*{Формула левых прямоугольников:}
$\displaystyle \int \limits_{x_0}^{x_1} f(x)\,dx = h(f_0+f_1+...+f_{n-1}),m=1$

% Please add the following required packages to your document preamble:
% \usepackage{graphicx}
\begin{table}[!h]
	\resizebox{\columnwidth}{!}{%
		\begin{tabular}{|c|c|c|c|c|c|c|}
			\hline
			n & $I_h$ & $I_0-I_h$ & $I_h-I_{2h}$ & $\dfrac{I_h-I_{2h}}{2^m-1}$ & $\dfrac{I_{2h}-I_{4h}}{I_h-I_{2h}}$ & $\log_2\left(\dfrac{I_{2h}-I_{4h}}{I_h-I_{2h}}\right)$ \\ \hline
			2    & 1.386178457532572 & 2.134230572759938    & 0                    & 0                    & 0                 & 0                  \\ \hline
			4    & 2.375413171808662 & 1.144995858483848    & 0.9892347142760902   & 0.9892347142760902   & 0                 & 0                  \\ \hline
			8    & 2.92956881756215  & 0.5908402127303605   & 0.5541556457534873   & 0.5541556457534873   & 1.785120700035501 & 0.8360216246496412 \\ \hline
			16   & 3.220476212290096 & 0.2999328180024143   & 0.2909073947279461   & 0.2909073947279461   & 1.90492113915402  & 0.9297312735682663 \\ \hline
			32   & 3.369319040994594 & 0.1510899892979163   & 0.1488428287044981   & 0.1488428287044981   & 1.954460266980635 & 0.9667702557898223 \\ \hline
			64   & 3.444583428661251 & 0.07582560163125907  & 0.07526438766665722  & 0.07526438766665722  & 1.977599676539144 & 0.9837504123843491 \\ \hline
			128  & 3.482426095748898 & 0.03798293454361223  & 0.03784266708764683  & 0.03784266708764683  & 1.988876404835275 & 0.9919536252558733 \\ \hline
			256  & 3.501400030714994 & 0.01900899957751578  & 0.01897393496609645  & 0.01897393496609645  & 1.994455401858705 & 0.9959948635922129 \\ \hline
			512  & 3.510900147497721 & 0.009508882794789564 & 0.00950011678272622  & 0.00950011678272622  & 1.997231760413324 & 0.9980017540130047 \\ \hline
			1024 & 3.515653493148001 & 0.004755537144509159 & 0.004753345650280405 & 0.004753345650280405 & 1.998616865189637 & 0.999001933979348  \\ \hline
		\end{tabular}%
	}
\end{table}
\subsubsection*{Формула правых прямоугольников:}
$\displaystyle\int \limits_{x_0}^{x_1} f(x)\,dx = h(f_1+f_2+...+f_{n}),m=1$
% Please add the following required packages to your document preamble:
% \usepackage{graphicx}
\begin{table}[!h]
	\resizebox{\columnwidth}{!}{%
		\begin{tabular}{|c|c|c|c|c|c|c|}
			\hline
			n & $I_h$ & $I_0-I_h$ & $I_h-I_{2h}$ & $\dfrac{I_h-I_{2h}}{2^m-1}$ & $\dfrac{I_{2h}-I_{4h}}{I_h-I_{2h}}$ & $\log_2\left(\dfrac{I_{2h}-I_{4h}}{I_h-I_{2h}}\right)$ \\ \hline
			2    & 6.256970537255301 & -2.73656150696279     & 0                     & 0                     & 0                 & 0                 \\ \hline
			4    & 4.810809211670027 & -1.290400181377517    & -1.446161325585273    & -1.446161325585273    & 0                 & 0                 \\ \hline
			8    & 4.147266837492832 & -0.6268578072003215   & -0.6635423741771955   & -0.6635423741771955   & 2.17945587480911  & 1.123967995231946 \\ \hline
			16   & 3.829325222255437 & -0.3089161919629269   & -0.3179416152373946   & -0.3179416152373946   & 2.086994411479461 & 1.061426737277019 \\ \hline
			32   & 3.673743545977264 & -0.1533345156847541   & -0.1555816762781728   & -0.1555816762781728   & 2.043567230044044 & 1.03108970646387  \\ \hline
			64   & 3.596795681152587 & -0.07638665086007634  & -0.07694786482467775  & -0.07694786482467775  & 2.021910245757418 & 1.015718956254445 \\ \hline
			128  & 3.558532221994566 & -0.03812319170205569  & -0.03826345915802065  & -0.03826345915802065  & 2.011001266427534 & 1.007913990180384 \\ \hline
			256  & 3.539453093837828 & -0.01904406354531796  & -0.01907912815673773  & -0.01907912815673773  & 2.005514027878052 & 1.003972057341313 \\ \hline
			512  & 3.529926679059137 & -0.009517648766627307 & -0.009526414778690651 & -0.009526414778690651 & 2.002760597766041 & 1.001989977287493 \\ \hline
			1024 & 3.52516675892871  & -0.004757728636199499 & -0.004759920130427808 & -0.004759920130427808 & 2.001381224401857 & 1.000995998912529 \\ \hline
		\end{tabular}%
	}
\end{table}

\subsubsection*{Формула центральных прямоугольников:}
$\displaystyle \int \limits_{x_0}^{x_1} f(x)\,dx = h(f_{\frac{1}{2}}+f_{\frac{3}{2}}+...+f_{n-\frac{1}{2}}),m=2$
% Please add the following required packages to your document preamble:
% \usepackage{graphicx}
\begin{table}[!h]
	\resizebox{\columnwidth}{!}{%
		\begin{tabular}{|c|c|c|c|c|c|c|}
			\hline
			n & $I_h$ & $I_0-I_h$ & $I_h-I_{2h}$ & $\dfrac{I_h-I_{2h}}{2^m-1}$ & $\dfrac{I_{2h}-I_{4h}}{I_h-I_{2h}}$ & $\log_2\left(\dfrac{I_{2h}-I_{4h}}{I_h-I_{2h}}\right)$ \\ \hline
			2    & 3.364647886084752 & 0.1557611442077582    & 0                     & 0                     & 0                 & 0                 \\ \hline
			4    & 3.483724463315637 & 0.03668456697687317   & 0.1190765772308851    & 0.03969219241029501   & 0                 & 0                 \\ \hline
			8    & 3.511383607018042 & 0.009025423274468203  & 0.02765914370240496   & 0.009219714567468321  & 4.305143301328282 & 2.106061260078529 \\ \hline
			16   & 3.518161869699092 & 0.002247160593417785  & 0.006778262681050418  & 0.002259420893683473  & 4.080565331250731 & 2.028769040454529 \\ \hline
			32   & 3.519847816327909 & 0.0005612139646009595 & 0.001685946628816826  & 0.0005619822096056085 & 4.020449144233771 & 2.007356680996974 \\ \hline
			64   & 3.520268762836543 & 0.0001402674559671802 & 0.0004209465086337794 & 0.0001403155028779265 & 4.005132705076283 & 2.00185004532653  \\ \hline
			128  & 3.52037396568109  & 3.506461142022133e-05 & 0.0001052028445469588 & 3.506761484898627e-05 & 4.001284475210969 & 2.000463202136774 \\ \hline
			256  & 3.520400264280449 & 8.766012061567352e-06 & 2.629859935865397e-05 & 8.766199786217991e-06 & 4.000321200084755 & 2.000115843791285 \\ \hline
			512  & 3.520406838798295 & 2.191494214986989e-06 & 6.574517846580363e-06 & 2.191505948860121e-06 & 4.000080305863463 & 2.000028963926997 \\ \hline
			1024 & 3.520408482419508 & 5.478730020769262e-07 & 1.643621212910062e-06 & 5.478737376366875e-07 & 4.000020074540201 & 2.000007240341731 \\ \hline
		\end{tabular}%
	}
\end{table}
\subsubsection*{Формула трапеций:}
$\displaystyle\int \limits_{x_0}^{x_1} f(x)\,dx = \dfrac{h}{2}(f_{0}+2f_{1}+...+2f_{n-1}+f_n),m=2$
% Please add the following required packages to your document preamble:
% \usepackage{graphicx}
\begin{table}[!h]
	\resizebox{\columnwidth}{!}{%
		\begin{tabular}{|c|c|c|c|c|c|c|}
			\hline
			n & $I_h$ & $I_0-I_h$ & $I_h-I_{2h}$ & $\dfrac{I_h-I_{2h}}{2^m-1}$ & $\dfrac{I_{2h}-I_{4h}}{I_h-I_{2h}}$ & $\log_2\left(\dfrac{I_{2h}-I_{4h}}{I_h-I_{2h}}\right)$ \\ \hline
			2    & 3.821574497393936 & -0.3011654671014261    & 0                      & 0                      & 0                 & 0                 \\ \hline
			4    & 3.593111191739345 & -0.07270216144683461   & -0.2284633056545915    & -0.07615443521819716   & 0                 & 0                 \\ \hline
			8    & 3.538417827527491 & -0.01800879723498072   & -0.05469336421185389   & -0.0182311214039513    & 4.177166808932112 & 2.062524756527754 \\ \hline
			16   & 3.524900717272766 & -0.004491686980256038  & -0.01351711025472468   & -0.004505703418241562  & 4.046232011219759 & 2.016579046500131 \\ \hline
			32   & 3.521531293485929 & -0.001122263193419126  & -0.003369423786836911  & -0.00112314126227897   & 4.011697877699747 & 2.004212960198181 \\ \hline
			64   & 3.520689554906919 & -0.0002805246144088613 & -0.0008417385790102649 & -0.000280579526336755  & 4.002933774045091 & 2.001057747464877 \\ \hline
			128  & 3.520479158871732 & -7.012857922150673e-05 & -0.0002103960351873546 & -7.01320117291182e-05  & 4.000734035984609 & 2.000264723229903 \\ \hline
			256  & 3.520426562276411 & -1.753198390108679e-05 & -5.259659532041994e-05 & -1.753219844013998e-05 & 4.000183546208953 & 2.000066198782552 \\ \hline
			512  & 3.520413413278429 & -4.382985919093585e-06 & -1.31489979819932e-05  & -4.382999327331068e-06 & 4.000045888853884 & 2.000016550810546 \\ \hline
			1024 & 3.520410126038356 & -1.095745845614005e-06 & -3.28724007347958e-06  & -1.09574669115986e-06  & 4.000011464959675 & 2.000004135104191 \\ \hline
		\end{tabular}%
	}
\end{table}


\subsubsection*{Формула Симпсона:}
$\displaystyle\int \limits_{x_0}^{x_1} f(x)\,dx = \dfrac{h}{3}(f_{0}+4f_{1}+2f_2+...+4f_{n-1}+f_n),m=4$
% Please add the following required packages to your document preamble:
% \usepackage{graphicx}
\begin{table}[!h]
	\resizebox{\columnwidth}{!}{%
		\begin{tabular}{|c|c|c|c|c|c|c|}
			\hline
			n & $I_h$ & $I_0-I_h$ & $I_h-I_{2h}$ & $\dfrac{I_h-I_{2h}}{2^m-1}$ & $\dfrac{I_{2h}-I_{4h}}{I_h-I_{2h}}$ & $\log_2\left(\dfrac{I_{2h}-I_{4h}}{I_h-I_{2h}}\right)$ \\ \hline
			2    & 3.471835303284339 & 0.04857372700817164   & 0                     & 0                     & 0                 & 0                 \\ \hline
			4    & 3.516956756521147 & 0.003452273771363146  & 0.0451214532368085    & 0.003008096882453899  & 0                 & 0                 \\ \hline
			8    & 3.520186706123539 & 0.0002223241689707223 & 0.003229949602392423  & 0.0002153299734928282 & 13.96970813519413 & 3.804229974166831 \\ \hline
			16   & 3.520395013854525 & 1.401643798537577e-05 & 0.0002083077309853465 & 1.388718206568977e-05 & 15.50566360218112 & 3.954723365764627 \\ \hline
			32   & 3.52040815222365  & 8.780688602882947e-07 & 1.313836912508748e-05 & 8.758912750058319e-07 & 15.85491540099803 & 3.986858274386117 \\ \hline
			64   & 3.520408975380582 & 5.491192789364163e-08 & 8.23156932394653e-07  & 5.48771288263102e-08  & 15.96095301884482 & 3.99647489146722  \\ \hline
			128  & 3.520409026860002 & 3.432508499656706e-09 & 5.147941939398493e-08 & 3.431961292932328e-09 & 15.99001974157529 & 3.999099814884794 \\ \hline
			256  & 3.52040903007797  & 2.145399413677751e-10 & 3.217968558288931e-09 & 2.14531237219262e-10  & 15.99748986402706 & 3.999773647199018 \\ \hline
			512  & 3.520409030279103 & 1.340749733458324e-11 & 2.011324440331919e-10 & 1.340882960221279e-11 & 15.9992515069219  & 3.999932507968077 \\ \hline
			1024 & 3.520409030291673 & 8.37108160567368e-13  & 1.257038917401587e-11 & 8.380259449343915e-13 & 16.00049459478556 & 4.000044596150995 \\ \hline
		\end{tabular}%
	}
\end{table}

\subsubsection*{Формула Гауса:}
$\displaystyle\int \limits_{x_0}^{x_1} f(x)\,dx = \dfrac{h}{18}(f_{k_{-1}}+2f_{k_0}+f_{k_1}),m=6$
\begin{table}[!h]
	\resizebox{\columnwidth}{!}{%
		\begin{tabular}{|c|c|c|c|c|c|c|}
			\hline
			n & $I_h$ & $I_0-I_h$ & $I_h-I_{2h}$ & $\dfrac{I_h-I_{2h}}{2^m-1}$ & $\dfrac{I_{2h}-I_{4h}}{I_h-I_{2h}}$ & $\log_2\left(\dfrac{I_{2h}-I_{4h}}{I_h-I_{2h}}\right)$ \\ \hline
			2    & 3.520415687285226 & -6.65699271618081e-06  & 0                      & 0                      & 0                 & 0                 \\ \hline
			4    & 3.520409153905599 & -1.236130890980291e-07 & -6.53337962708278e-06  & -1.037044385251235e-07 & 0                 & 0                 \\ \hline
			8    & 3.520409032380652 & -2.088141659584153e-09 & -1.21524947438445e-07  & -1.928967419657857e-09 & 53.76163302100647 & 5.748505055837482 \\ \hline
			16   & 3.520409030325908 & -3.339772902677396e-11 & -2.054743930557379e-09 & -3.26149830247203e-11  & 59.1435972294025  & 5.886150088551767 \\ \hline
			32   & 3.520409030293036 & -5.253575352526241e-13 & -3.287237149152133e-11 & -5.217836744685926e-13 & 62.50671422009673 & 5.96593926149044  \\ \hline
			64   & 3.520409030292518 & -7.993605777301127e-15 & -5.173639294753229e-13 & -8.212125864687665e-15 & 63.5381974248927  & 5.989552256341715 \\ \hline
			128  & 3.520409030292511 & -4.440892098500626e-16 & -7.549516567451064e-15 & -1.19833596308747e-16  & 68.52941176470588 & 6.098651398291303 \\ \hline
			256  & 3.520409030292511 & -8.881784197001252e-16 & 4.440892098500626e-16  & 7.049035076985121e-18  & -17               & 4.087462841250339 \\ \hline
			512  & 3.520409030292511 & -8.881784197001252e-16 & 0                      & 0                      & Inf               & Inf               \\ \hline
			1024 & 3.52040903029251  & 0                      & -8.881784197001252e-16 & -1.409807015397024e-17 & 0                 & -Inf              \\ \hline
		\end{tabular}%
	}
\end{table}

\subsection{$\displaystyle\int\limits_0^{2\pi} \sqrt{7-5\cos x}\,dx=16.01001897674756$}
\subsubsection*{Формула левых прямоугольников:}
$\displaystyle \int \limits_{x_0}^{x_1} f(x)\,dx = h(f_0+f_1+...+f_{n-1}),m=1$
% Please add the following required packages to your document preamble:
% \usepackage{graphicx}
\begin{table}[!h]
	\resizebox{\columnwidth}{!}{%
		\begin{tabular}{|c|c|c|c|c|c|c|}
			\hline
			n & $I_h$ & $I_0-I_h$ & $I_h-I_{2h}$ & $\dfrac{I_h-I_{2h}}{2^m-1}$ & $\dfrac{I_{2h}-I_{4h}}{I_h-I_{2h}}$ & $\log_2\left(\dfrac{I_{2h}-I_{4h}}{I_h-I_{2h}}\right)$ \\ \hline
			2    & 15.32567912356367 & 0.6843398531838876     & 0                     & 0                     & 0                   & 0                   \\ \hline
			4    & 15.97471244384792 & 0.03530653289964292    & 0.6490333202842447    & 0.6490333202842447    & 0                   & 0                   \\ \hline
			8    & 16.00965865537168 & 0.00036032137587938    & 0.03494621152376354   & 0.03494621152376354   & 18.57235139330911   & 4.215084577296379   \\ \hline
			16   & 16.01001885699137 & 1.197561907417821e-07  & 0.0003602016196886382 & 0.0003602016196886382 & 97.01847413671095   & 6.600187584534389   \\ \hline
			32   & 16.01001897674752 & 3.907985046680551e-14  & 1.197561516619317e-07 & 1.197561516619317e-07 & 3007.792206829404   & 11.55448918660725   \\ \hline
			64   & 16.01001897674756 & 0                      & 3.907985046680551e-14 & 3.907985046680551e-14 & 3064396.363636364   & 21.54717148371147   \\ \hline
			128  & 16.01001897674756 & -3.552713678800501e-15 & 3.552713678800501e-15 & 3.552713678800501e-15 & 11                  & 3.459431618637297   \\ \hline
			256  & 16.01001897674755 & 7.105427357601002e-15  & -1.06581410364015e-14 & -1.06581410364015e-14 & -0.3333333333333333 & -1.584962500721156  \\ \hline
			512  & 16.01001897674753 & 2.486899575160351e-14  & -1.77635683940025e-14 & -1.77635683940025e-14 & 0.6                 & -0.7369655941662062 \\ \hline
			1024 & 16.01001897674757 & -7.105427357601002e-15 & 3.197442310920451e-14 & 3.197442310920451e-14 & -0.5555555555555556 & -0.84799690655495   \\ \hline
		\end{tabular}%
	}
\end{table}
\subsubsection*{Формула правых прямоугольников:}
$\displaystyle\int \limits_{x_0}^{x_1} f(x)\,dx = h(f_1+f_2+...+f_{n}),m=1$
% Please add the following required packages to your document preamble:
% \usepackage{graphicx}
\begin{table}[!h]
	\resizebox{\columnwidth}{!}{%
		\begin{tabular}{|c|c|c|c|c|c|c|}
			\hline
			n & $I_h$ & $I_0-I_h$ & $I_h-I_{2h}$ & $\dfrac{I_h-I_{2h}}{2^m-1}$ & $\dfrac{I_{2h}-I_{4h}}{I_h-I_{2h}}$ & $\log_2\left(\dfrac{I_{2h}-I_{4h}}{I_h-I_{2h}}\right)$ \\ \hline
			2    & 15.32567912356367 & 0.6843398531838876     & 0                     & 0                     & 0                   & 0                   \\ \hline
			4    & 15.97471244384792 & 0.03530653289963936    & 0.6490333202842482    & 0.6490333202842482    & 0                   & 0                   \\ \hline
			8    & 16.00965865537168 & 0.00036032137587938    & 0.03494621152375998   & 0.03494621152375998   & 18.5723513933111    & 4.215084577296533   \\ \hline
			16   & 16.01001885699137 & 1.197561942944958e-07  & 0.0003602016196850855 & 0.0003602016196850855 & 97.01847413765799   & 6.600187584548471   \\ \hline
			32   & 16.01001897674752 & 3.907985046680551e-14  & 1.197561552146453e-07 & 1.197561552146453e-07 & 3007.792117569881   & 11.5544891437937    \\ \hline
			64   & 16.01001897674756 & 0                      & 3.907985046680551e-14 & 3.907985046680551e-14 & 3064396.454545455   & 21.54717152651079   \\ \hline
			128  & 16.01001897674756 & -3.552713678800501e-15 & 3.552713678800501e-15 & 3.552713678800501e-15 & 11                  & 3.459431618637297   \\ \hline
			256  & 16.01001897674755 & 7.105427357601002e-15  & -1.06581410364015e-14 & -1.06581410364015e-14 & -0.3333333333333333 & -1.584962500721156  \\ \hline
			512  & 16.01001897674753 & 2.486899575160351e-14  & -1.77635683940025e-14 & -1.77635683940025e-14 & 0.6                 & -0.7369655941662062 \\ \hline
			1024 & 16.01001897674757 & -7.105427357601002e-15 & 3.197442310920451e-14 & 3.197442310920451e-14 & -0.5555555555555556 & -0.84799690655495   \\ \hline
		\end{tabular}%
	}
\end{table}
\newpage
\subsubsection*{Формула центральных прямоугольников:}
$\displaystyle \int \limits_{x_0}^{x_1} f(x)\,dx = h(f_{\frac{1}{2}}+f_{\frac{3}{2}}+...+f_{n-\frac{1}{2}}),m=2$
\begin{table}[!h]
	\resizebox{\columnwidth}{!}{%
		\begin{tabular}{|c|c|c|c|c|c|c|}
			\hline
			n & $I_h$ & $I_0-I_h$ & $I_h-I_{2h}$ & $\dfrac{I_h-I_{2h}}{2^m-1}$ & $\dfrac{I_{2h}-I_{4h}}{I_h-I_{2h}}$ & $\log_2\left(\dfrac{I_{2h}-I_{4h}}{I_h-I_{2h}}\right)$ \\ \hline
			2    & 16.62374576413216 & -0.6137267873846035    & 0                      & 0                      & 0                   & 0                  \\ \hline
			4    & 16.04460486689545 & -0.03458589014788771   & -0.5791408972367158    & -0.1930469657455719    & 0                   & 0                  \\ \hline
			8    & 16.01037905861105 & -0.000360081863490791  & -0.03422580828439692   & -0.01140860276146564   & 16.92117516770928   & 4.080757861321133  \\ \hline
			16   & 16.01001909650367 & -1.197561090293675e-07 & -0.0003599621073817616 & -0.0001199873691272539 & 95.08169771908346   & 6.571095758254331  \\ \hline
			32   & 16.01001897674759 & -3.197442310920451e-14 & -1.197560770549444e-07 & -3.991869235164813e-08 & 3005.794079381959   & 11.55353046130478  \\ \hline
			64   & 16.01001897674756 & 3.552713678800501e-15  & -3.552713678800501e-14 & -1.1842378929335e-14   & 3370833.9           & 21.68467410867531  \\ \hline
			128  & 16.01001897674757 & -7.105427357601002e-15 & 1.06581410364015e-14   & 3.552713678800501e-15  & -3.333333333333333  & 1.736965594166206  \\ \hline
			256  & 16.01001897674757 & -1.06581410364015e-14  & 3.552713678800501e-15  & 1.1842378929335e-15    & 3                   & 1.584962500721156  \\ \hline
			512  & 16.01001897674755 & 1.06581410364015e-14   & -2.131628207280301e-14 & -7.105427357601002e-15 & -0.1666666666666667 & -2.584962500721156 \\ \hline
			1024 & 16.01001897674757 & -7.105427357601002e-15 & 1.77635683940025e-14   & 5.921189464667502e-15  & -1.2                & 0.2630344058337938 \\ \hline
		\end{tabular}%
	}
\end{table}
\subsubsection*{Формула трапеций:}
$\displaystyle\int \limits_{x_0}^{x_1} f(x)\,dx = \dfrac{h}{2}(f_{0}+2f_{1}+...+2f_{n-1}+f_n),m=2$
% Please add the following required packages to your document preamble:
% \usepackage{graphicx}
\begin{table}[!h]
	\resizebox{\columnwidth}{!}{%
		\begin{tabular}{|c|c|c|c|c|c|c|}
			\hline
			n & $I_h$ & $I_0-I_h$ & $I_h-I_{2h}$ & $\dfrac{I_h-I_{2h}}{2^m-1}$ & $\dfrac{I_{2h}-I_{4h}}{I_h-I_{2h}}$ & $\log_2\left(\dfrac{I_{2h}-I_{4h}}{I_h-I_{2h}}\right)$ \\ \hline
			2    & 15.32567912356367 & 0.6843398531838876     & 0                     & 0                      & 0                   & 0                   \\ \hline
			4    & 15.97471244384792 & 0.03530653289963936    & 0.6490333202842482    & 0.2163444400947494     & 0                   & 0                   \\ \hline
			8    & 16.00965865537168 & 0.00036032137587938    & 0.03494621152375998   & 0.01164873717458666    & 18.5723513933111    & 4.215084577296533   \\ \hline
			16   & 16.01001885699137 & 1.197561942944958e-07  & 0.0003602016196850855 & 0.0001200672065616952  & 97.01847413765799   & 6.600187584548471   \\ \hline
			32   & 16.01001897674752 & 3.907985046680551e-14  & 1.197561552146453e-07 & 3.991871840488178e-08  & 3007.792117569881   & 11.5544891437937    \\ \hline
			64   & 16.01001897674756 & 0                      & 3.907985046680551e-14 & 1.30266168222685e-14   & 3064396.454545455   & 21.54717152651079   \\ \hline
			128  & 16.01001897674756 & -3.552713678800501e-15 & 3.552713678800501e-15 & 1.1842378929335e-15    & 11                  & 3.459431618637297   \\ \hline
			256  & 16.01001897674755 & 7.105427357601002e-15  & -1.06581410364015e-14 & -3.552713678800501e-15 & -0.3333333333333333 & -1.584962500721156  \\ \hline
			512  & 16.01001897674753 & 2.486899575160351e-14  & -1.77635683940025e-14 & -5.921189464667502e-15 & 0.6                 & -0.7369655941662062 \\ \hline
			1024 & 16.01001897674757 & -7.105427357601002e-15 & 3.197442310920451e-14 & 1.06581410364015e-14   & -0.5555555555555556 & -0.84799690655495   \\ \hline
		\end{tabular}%
	}
\end{table}

\subsubsection*{Формула Симпсона:}
$\displaystyle\int \limits_{x_0}^{x_1} f(x)\,dx = \dfrac{h}{3}(f_{0}+4f_{1}+2f_2+...+4f_{n-1}+f_n),m=4$

% Please add the following required packages to your document preamble:
% \usepackage{graphicx}
\begin{table}[!h]
	\resizebox{\columnwidth}{!}{%
		\begin{tabular}{|c|c|c|c|c|c|c|}
			\hline
			n & $I_h$ & $I_0-I_h$ & $I_h-I_{2h}$ & $\dfrac{I_h-I_{2h}}{2^m-1}$ & $\dfrac{I_{2h}-I_{4h}}{I_h-I_{2h}}$ & $\log_2\left(\dfrac{I_{2h}-I_{4h}}{I_h-I_{2h}}\right)$ \\ \hline
			2    & 17.47231687264599 & -1.462297895898427     & 0                      & 0                      & 0                 & 0                  \\ \hline
			4    & 16.19105688394266 & -0.1810379071951047    & -1.281259988703322     & -0.08541733258022148   & 0                 & 0                  \\ \hline
			8    & 16.02130739254627 & -0.01128841579870965   & -0.1697494913963951    & -0.01131663275975967   & 7.547945965336377 & 2.916084094928252  \\ \hline
			16   & 16.01013892419792 & -0.0001199474503650322 & -0.01116846834834462   & -0.0007445645565563078 & 15.19899471457563 & 3.925903999591842  \\ \hline
			32   & 16.01001901666623 & -3.991867458807974e-08 & -0.0001199075316904441 & -7.993835446029607e-06 & 93.14234219396143 & 6.541365256078531  \\ \hline
			64   & 16.01001897674757 & -7.105427357601002e-15 & -3.991866748265238e-08 & -2.661244498843492e-09 & 3003.795949415216 & 11.55257109725949  \\ \hline
			128  & 16.01001897674756 & 3.552713678800501e-15  & -1.06581410364015e-14  & -7.105427357601002e-16 & 3745368.666666667 & 21.8366763033337   \\ \hline
			256  & 16.01001897674756 & -3.552713678800501e-15 & 7.105427357601002e-15  & 4.736951571734001e-16  & -1.5              & 0.5849625007211562 \\ \hline
			512  & 16.01001897674756 & -3.552713678800501e-15 & 0                      & 0                      & Inf               & Inf                \\ \hline
			1024 & 16.01001897674754 & 1.77635683940025e-14   & -2.131628207280301e-14 & -1.4210854715202e-15   & 0                 & -Inf               \\ \hline
		\end{tabular}%
	}
\end{table}
\subsubsection*{Формула Гауса:}
$\displaystyle\int \limits_{x_0}^{x_1} f(x)\,dx = \dfrac{h}{18}(f_{k_{-1}}+2f_{k_0}+f_{k_1}),m=6$
\begin{table}[!h]
	\resizebox{\columnwidth}{!}{%
		\begin{tabular}{|c|c|c|c|c|c|c|}
			\hline
			n & $I_h$ & $I_0-I_h$ & $I_h-I_{2h}$ & $\dfrac{I_h-I_{2h}}{2^m-1}$ & $\dfrac{I_{2h}-I_{4h}}{I_h-I_{2h}}$ & $\log_2\left(\dfrac{I_{2h}-I_{4h}}{I_h-I_{2h}}\right)$ \\ \hline
			2    & 16.00843988196752 & 0.001579094780041146   & 0                      & 0                      & 0                 & 0                 \\ \hline
			4    & 16.01061705530846 & -0.0005980785608983297 & 0.002177173340939476   & 3.455830699903929e-05  & 0                 & 0                 \\ \hline
			8    & 16.01002700136552 & -8.024617962121283e-06 & -0.0005900539429362084 & -9.365935602162038e-06 & -3.68978695423251 & 1.883537518468476 \\ \hline
			16   & 16.01001897943656 & -2.689002798206275e-09 & -8.021928959323077e-06 & -1.273322057035409e-07 & 73.55511946418427 & 6.200753851983465 \\ \hline
			32   & 16.01001897674756 & 3.552713678800501e-15  & -2.689006350919954e-09 & -4.268264049079291e-11 & 2983.231689496993 & 11.54266031205299 \\ \hline
			64   & 16.01001897674756 & 3.552713678800501e-15  & 0                      & 0                      & -Inf              & Inf               \\ \hline
			128  & 16.01001897674756 & 3.552713678800501e-15  & 0                      & 0                      & NaN               & NaN               \\ \hline
			256  & 16.01001897674756 & -3.552713678800501e-15 & 7.105427357601002e-15  & 1.127845612317619e-16  & 0                 & -Inf              \\ \hline
			512  & 16.01001897674756 & 0                      & -3.552713678800501e-15 & -5.639228061588096e-17 & -2                & 1                 \\ \hline
			1024 & 16.01001897674757 & -1.4210854715202e-14   & 1.4210854715202e-14    & 2.255691224635239e-16  & -0.25             & -2                \\ \hline
		\end{tabular}%
	}
\end{table}
\subsection{$\displaystyle \int \limits_0^\infty \dfrac{5+3x}{1+2x^5+4x^6}\,dx$}
\subsubsection*{Первый способ:}
$\displaystyle\int\limits_0^\infty \dfrac{5+3x}{1+2x^5+4x^6}\,dx=\begin{bmatrix}t=\dfrac{1}{x+1}&x=\dfrac{1}{t}-1&\infty=0\\dx=\dfrac{1}{t^2}dt&0=1\end{bmatrix}=$
\newpage
$\displaystyle=\int\limits_0^{1} \dfrac{5+3\left(\dfrac{1}{t}-1\right)}{1+2\left(\dfrac{1}{t}-1\right)^5+4\left(\dfrac{1}{t}-1\right)^6}\dfrac{1}{t^2}\,dt\\$\\
$$\int\limits_0^{1} \dfrac{5+3\left(\dfrac{1}{t}-1\right)}{1+2\left(\dfrac{1}{t}-1\right)^5+4\left(\dfrac{1}{t}-1\right)^6}\dfrac{1}{t^2}\,dt=4.807851149764657\\$$

% Please add the following required packages to your document preamble:
% \usepackage{graphicx}
\begin{table}[!h]
	\resizebox{\columnwidth}{!}{%
		\begin{tabular}{|c|c|c|c|c|c|c|}
			\hline
			n & $I_h$ & $I_0-I_h$ & $I_h-I_{2h}$ & $\dfrac{I_h-I_{2h}}{2^m-1}$ & $\dfrac{I_{2h}-I_{4h}}{I_h-I_{2h}}$ & $\log_2\left(\dfrac{I_{2h}-I_{4h}}{I_h-I_{2h}}\right)$ \\ \hline
			2   & 4.673172432257312 & 0.1346787175073452     & 0                      & 0                      & 0                   & 0                  \\ \hline
			4   & 4.816906551756611 & -0.009055401991953538  & 0.1437341194992987     & 0.002281493960306329   & 0                   & 0                  \\ \hline
			8   & 4.807771305113582 & 7.984465107480077e-05  & -0.009135246643028339  & -0.0001450039149687038 & -15.73401629051701  & 3.975815077326897  \\ \hline
			16  & 4.80785133734577  & -1.875811133089655e-07 & 8.003223218810973e-05  & 1.270352891874758e-06  & -114.1445939125705  & 6.834718722397196  \\ \hline
			32  & 4.807851149769121 & -4.463984737412829e-12 & -1.875766493242281e-07 & -2.977407132130605e-09 & -426.6641529019596  & 8.736957094307082  \\ \hline
			64  & 4.807851149764726 & -6.927791673660977e-14 & -4.39470682067622e-12  & -6.975725112184476e-14 & 42682.40339531124   & 15.3813537944389   \\ \hline
			128 & 4.807851149764657 & 0                      & -6.927791673660977e-14 & -1.099649472009679e-15 & 63.43589743589744   & 5.987227566114122  \\ \hline
			256 & 4.807851149764656 & 8.881784197001252e-16  & -8.881784197001252e-16 & -1.409807015397024e-17 & 78                  & 6.285402218862249  \\ \hline
			512 & 4.807851149764659 & -1.77635683940025e-15  & 2.664535259100376e-15  & 4.229421046191072e-17  & -0.3333333333333333 & -1.584962500721156 \\ \hline
		\end{tabular}%
	}
\end{table}

\subsubsection*{Второй способ:} 

$\displaystyle\int\limits_0^\infty \dfrac{5+3x}{1+2x^5+4x^6}\,dx=\displaystyle\int\limits_0^M \dfrac{5+3x}{1+2x^5+4x^6}\,dx$\\
$M:\left|\displaystyle\int\limits_0^M \dfrac{5+3x}{1+2x^5+4x^6}\,dx\right|<1\cdot10^{-9}$,~$x\rightarrow\infty\sim\dfrac{3}{4x^5}$\\
$\displaystyle\int \limits_M^{\infty}f(x)\,dx=\int \limits_M^{\infty}\dfrac{3}{4x^5}\,dx=-\dfrac{3}{16x^4}\bigg|_M^\infty=\dfrac{3}{16M^4}<10^{-9}$\\

$M\approx118$\\
$$\displaystyle\int\limits_0^{118} \dfrac{5+3x}{1+2x^5+4x^6}\,dx=4.807851148789965$$\\

\subsubsection*{Третий способ:}

$\displaystyle\int\limits_0^\infty \dfrac{5+3x}{1+2x^5+4x^6}\,dx=\int \limits_0^1 f_4(x)\,dx+\int \limits_1^2 f_4(x)\,dx+...+\int \limits_{128}^{256} f_4(x)\,dx$
$$\displaystyle\int\limits_0^{\infty} \dfrac{5+3x}{1+2x^5+4x^6}\,dx= 4.807851149061087$$
% Please add the following required packages to your document preamble:
% \usepackage{graphicx}
% Please add the following required packages to your document preamble:
% \usepackage{graphicx}
% Please add the following required packages to your document preamble:
% \usepackage{graphicx}
\begin{table}[!h]
	\centering
	\begin{tabular}{|c|c|c|c|}
		\hline
		k  & $M$  & $I_k$                 & $\displaystyle\sum I_k$ \\ \hline
		1  & 2    & 4.516657267227301     & 4.516657267227301       \\ \hline
		2  & 4    & 0.2749665927291963    & 4.791623859956498       \\ \hline
		3  & 8    & 0.01534007172099053   & 4.806963931677488       \\ \hline
		4  & 16   & 0.0008363651882959215 & 4.807800296865784       \\ \hline
		5  & 32   & 4.782921578927475e-05 & 4.807848126081574       \\ \hline
		6  & 64   & 2.839720753999986e-06 & 4.807850965802327       \\ \hline
		7  & 128  & 1.726245306938752e-07 & 4.807851138426858       \\ \hline
		8  & 256  & 1.063422974868193e-08 & 4.807851149061087       \\ \hline
		9  & 512  & 6.5975392864428e-10   & 4.807851149720841       \\ \hline
		10 & 1024 & 4.108119477864649e-11 & 4.807851149761922       \\ \hline
	\end{tabular}
\end{table}
\lstset{language=Matlab,%
	%basicstyle=\color{red},
	breaklines=true,%
	morekeywords={matlab2tikz},
	keywordstyle=\color{blue},%
	morekeywords=[2]{1}, keywordstyle=[2]{\color{black}},
	identifierstyle=\color{black},%
	stringstyle=\color{mylilas},
	commentstyle=\color{mygreen},%
	showstringspaces=false,%without this there will be a symbol in the places where there is a space
	numbers=left,%
	numberstyle={\small \color{black}},% size of the numbers
	numbersep=1pt, % this defines how far the numbers are from the text
	emph=[1]{for,end,break},emphstyle=[1]\color{red}, %some words to emphasise
	%emph=[2]{word1,word2}, emphstyle=[2]{style},    
}
\section{Листинг функций и сценариев}
\subsection*{f1.m}
\lstinputlisting{../f1.m}
\subsection*{f2a.m}
\lstinputlisting{../f2a.m}
\subsection*{f3.m}
\lstinputlisting{../f3.m}
\subsection*{f4.m}
\lstinputlisting{../f4.m}
\subsection*{f4a.m}
\lstinputlisting{../f4a.m}

\subsection*{qleft.m}
\lstinputlisting{../qleft.m}
\subsection*{qright.m}
\lstinputlisting{../qright.m}
\subsection*{qcenter.m}
\lstinputlisting{../qcenter.m}
\subsection*{qtrap.m}
\lstinputlisting{../qtrap.m}
\subsection*{qsimp.m}
\lstinputlisting{../qsimp.m}
\subsection*{qgaus.m}
\lstinputlisting{../qgaus.m}

\subsection*{wleft.m}
\lstinputlisting{../wleft.m}
\subsection*{wright.m}
\lstinputlisting{../wright.m}
\subsection*{wcenter.m}
\lstinputlisting{../wcenter.m}
\subsection*{wtrap.m}
\lstinputlisting{../wtrap.m}
\subsection*{wsimp.m}
\lstinputlisting{../wsimp.m}
\subsection*{wgaus.m}
\lstinputlisting{../wgaus.m}

\subsection*{w2aleft.m}
\lstinputlisting{../w2a_left.m}
\subsection*{w2aright.m}
\lstinputlisting{../w2a_right.m}
\subsection*{w2acenter.m}
\lstinputlisting{../w2a_center.m}
\subsection*{w2atrap.m}
\lstinputlisting{../w2a_trap.m}
\subsection*{w2asimp.m}
\lstinputlisting{../w2a_simp.m}
\subsection*{w2agaus.m}
\lstinputlisting{../w2a_gaus.m}

\subsection*{w3left.m}
\lstinputlisting{../w3_left.m}
\subsection*{w3right.m}
\lstinputlisting{../w3_right.m}
\subsection*{w3center.m}
\lstinputlisting{../w3_center.m}
\subsection*{w3trap.m}
\lstinputlisting{../w3_trap.m}
\subsection*{w3simp.m}
\lstinputlisting{../w3_simp.m}
\subsection*{w3gaus.m}
\lstinputlisting{../w3_gaus.m}

\subsection*{w4.m}
\lstinputlisting{../w4.m}
\subsection*{w4sum.m}
\lstinputlisting{../w4_sum.m}
\subsection*{w4a.m}
\lstinputlisting{../w4a.m}
\end{document}